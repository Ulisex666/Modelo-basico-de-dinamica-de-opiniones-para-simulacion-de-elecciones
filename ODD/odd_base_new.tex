 \documentclass[11pt, letterpaper]{article}
\usepackage[legalpaper, margin=3 cm]{geometry}
\usepackage[utf8]{inputenc}
\usepackage[T1]{fontenc}
\usepackage[spanish]{babel}
\spanishdecimal{.}
%\usepackage{lmodern}
\usepackage{graphicx}
%\usepackage{wrapfig}
%\usepackage{rotating}
%\usepackage{subfig}
\usepackage{amsmath}
%\usepackage{textcomp}
\usepackage{amssymb}
\usepackage{hyperref}
%\usepackage{longtable}
%\usepackage{minted}
\usepackage{makecell}
\usepackage{lipsum}
\usepackage[spanish]{babel}
\usepackage[round]{natbib}

\begin{document}

\section{Protocolo ODD para modelo base de dinámica de
opiniones}\label{protocolo-odd-para-modelo-base-de-dinuxe1mica-de-opiniones}

A continuación se encuentra la descripción del modelo base de dinámica
de opiniones, siguiendo el protocolo ODD.

\subsection{Propósito y patrones}\label{propuxf3sito-y-patrones}

El propósito del modelo es la simulación de la elección presidencial
mexicana llevada a cabo en el 2024, mediante el uso de dinámica de
opiniones. En este contexto, \emph{una opinión es la posición que se
tiene sobre un tema, y se formaliza como un número que cambia entre dos
extremos}. Se espera que las personas modifiquen sus opiniones después
de interactuar con sus pares a través del efecto de la influencia
social. Se consideran tres tipos de influencia diferentes entre los
agentes durante de una interacción.

\begin{itemize}
\item
  \textbf{Influencia positiva.} Después de una interacción, dos agentes
  siempre tendrán una opinión más cercana a la del otro. Esto toma como
  base teorías cognitivas que hacen hincapié en el papel del aprendizaje
  social y de la presión social para seguir las normas de un grupo.
\item
  \textbf{Confianza acotada.} Un agente es influenciado a tomar una
  opinión más cercana a la del otro agente en la interacción solamente
  si sus opiniones son suficientemente similares. Qué tan parecida debe
  ser la opinión de otro agente está determinado por una cota o límite
  de confianza. La base teórica principal es el sesgo de confirmación,
  la tendencia de preferir información que este de acuerdo con lo que ya
  se opina y evadir aquella que contradiga nuestras creencias.
\item
  \textbf{Influencia negativa.} Si dos agentes con opiniones muy
  disimilares interactúan entre ellos, se influencian en sentido
  opuesto, tomando opiniones todavía más dispares. Esto toma como base
  teórica efectos como la xenofobia o el rechazo a grupos percibidos
  como externos, aunque la evidencia empírica de este efecto es mixta.
\end{itemize}

La opinión es convertida a voto dependiendo de su valor, tomando como
referencia los dos candidatos principales en la elección real. De esta
forma, el principal patrón a evaluar es la evolución del voto durante
todo el periodo de la simulación, con el objetivo de replicar la
distribución de voto observada en la elección real. Se utiliza una serie
de encuestas de opinión realizadas durante el periodo electoral como
refencia.

\subsection{Entidades, variables de estado y
escalas}\label{entidades-variables-de-estado-y-escalas}

Las únicas entidades del modelo son las parcelas, indicando un votante
dentro de las encuestas de opinión tomadas como referencia. Cada una se
encuentra definida por una variable de estado: su opinión en un momento
dado. Esta es un valor real en el rango \emph{{[}-1,1{]}}, donde 1
indica una opinión completamente a favor del candidato A y -1 una
opinión completamente a favor del candidato B. Cuando se activa el
parámetro \emph{local-interactions?} cada agente puede comunicarse
únicamente con sus cuatro vecinos más cercanos, por lo que esto se
vuelve otra característica que define su comportamiento.

Se utiliza un mundo cerrado de tamaño \emph{107x10}, con cada parcela
representando un votante de la encuestas. De esta forma, se tiene 1070
agentes dentro del modelo. El alcance temporal del modelo se da en base
al número de días desde la primera encuestas hasta el día de la
elección, con un total de 260 días. Cada tick representa un día, con el
modelo ejecutándose un total de 260 ticks.

\subsection{Descripción general y
\emph{scheduling}}\label{descripciuxf3n-general-y-scheduling}

El modelo realiza las siguientes acciones por cada paso de tiempo.

\textbf{Selección de agentes iniciales}. Se selecciona a una cantidad de
parcelas definida por el parámetro \emph{agents-updated-per-tick} de
manera aleatoria de entre todas las presentes en el modelo.

\textbf{Selección de agentes para interacción.} Cada uno de los \emph{n}
agentes iniciales selecciona a otro agente en el modelo para
interactuar. Si se tiene activado el parámetro
\emph{spatial-interactions?}, cada agente inicial debe seleccionar a uno
de sus vecinos. En caso contrario, selecciona un agente al azar de todos
los presentes en el modelo, permitiendo repeticiones.

\textbf{Interacción entre agentes.} Los agentes cambian su opinión de
acuerdo al tipo de inlfuencia seleccionado en la simulación, detallados
en la sección de submodelos. Si se cumplen las condiciones adecuadas,
ambos agentes presentes en la simulación modifican el valor de su
opinión.

\textbf{Actualización de las preferencias}. Al terminar las
interacciones entre los seleccionados, se actualizan las preferencias de
voto del modelo de acuerdo al cambio de las opiniones.

\textbf{Visualización.} El color de los agentes es modificado de acuerdo
al valor de su opinión, y se actualizan las gráficas y visualizaciones
en el modelo.

\textbf{Terminación}. Al llegar al tick 260, el modelo actualiza las
preferencias y las visualizaciones por una última vez, terminando su
ejecución.

\subsection{Conceptos de diseño}\label{conceptos-de-diseuxf1o}

\begin{itemize}
\item
  \textbf{Principios básicos.} La suposición básica en todo modelo de
  dinámica de opinión es que la influencia social tiene un papel
  fundamental a a la hora de formar y modificar opiniones. En este caso
  se esta consideran tres diferentes tipos de influencia al interactuar
  con otros agentes, cada una con diferentes justificaciones en ciencias
  sociales y psicología. Al activar las interacciones espaciales, se
  esta asumiendo que el efecto de la influencia social se da únicamente
  por aquellos considerados cercanos.
\item
  \textbf{Emergencia.} Los patrones principales a buscar son la
  distribución de opiniones a escala global en el sistema y la
  distribución de la preferencia de votos emergente de esta. Estos
  patrones emergen de la interacción entre pares de los agentes del
  sistema y del tipo de influencia seleccionado.
\item
  \textbf{Adaptación.} Los agentes adaptan su opinión después de
  interactuar con otros agentes de acuerdo al tipo de influencia
  seleccionado en la simulación. Para una influencia positiva, el agente
  busca parecerse más a cualquier otra persona después de una
  interacción. En el caso de confianza acotada, busca solamente
  parecerse a aquellos similares a sí mismo. Para la influencia
  negativa, busca alejar su opinión de aquellas que sean muy
  disimilares.
\item
  \textbf{Objetivos.} Los tres modelos de influencia asumen que un
  agente busca ajustar su opinión de acuerdo a las interacciones con
  otros. Se difiere en qué agentes son considerados suficientemente
  importantes como para afectar la opinión, y el sentido en que se
  afecta esta opinión.
\item
  \textbf{Aprendizaje.} Este concepto no se utiliza en el modelo.
\item
  \textbf{Predicción.} Este concepto no se utiliza en el modelo.
\item
  \textbf{Percepciones.} En la interacción entre agentes, se asume que
  cada agente es completamente consciente de la opinión del otro.
\item
  \textbf{Interacciones.} La interacción a modelar es el compartir
  opinión entre pares y cómo estas opiniones son modificadas de acuerdo
  al tipo de influencia considerado. Las interacciones se dan de forma
  directa, donde la opinión de un agente afecta la del otro de acuerdo
  al tipo de interacción.
\item
  \textbf{Estocásticidad.} Las opiniones iniciales de cada agente se dan
  de forma aleatoria, siguiendo una distribución uniforme. Los agentes a
  interactuar en cada paso de tiempo se seleccionan de manera aleatoria.
\item
  \textbf{Colectivos.} Este concepto no se utiliza en el modelo.
\item
  \textbf{Observaciones.} Los agentes cambian su color de acuerdo a la
  opinión que tengan en ese momento, con valores entre -1 y 1. Aquellos
  con valores cercanos a -1 tomarán un color azul intenso, mientras que
  aquellos con valores cercanos a 1 tomarán colores rojos.También se
  tiene un histograma de las opiniones en todo el sistema, junto a una
  gráfica que muestra su evolución cada 10 pasos de tiempo. Para la
  visualización de las preferencias de voto se cuenta con dos gráficos,
  uno que indica el porcentaje de preferencia por A mientras que la
  segunda muestra las preferencias por B, siendo actualizadas cada paso
  de tiempo.
\end{itemize}

\subsection{Inicialización}\label{inicializaciuxf3n}

El modelo se inicializa asignando las opiniones iniciales a los agentes
de acuerdo al parámetro \emph{percent-option-B}, indicando el porcentaje
de agentes con preferencia por la opción B. Para la comparación con las
encuestas de opinión utilizadas, este parámetro debe tomar el valor de
39. Las opiniones iniciales para los agentes con preferencia por A se
generan con una distribución uniforme con valores entre
\emph{{[}0,1{]}}, mientras que las opiniones para agentes con
preferencia por B toman valores de una distribución uniforme entre
\emph{{[}-1,0{]}}.

Posteriormente, se actualiza el color de los agentes de acuerdo a la
opinión asignada, y se actualizan las preferencias de voto por la opción
A y por la opción B.

\subsection{Datos de entrada}\label{datos-de-entrada}

El modelo no necesita del uso de datos de entrada.

\subsection{Submodelos}\label{submodelos}

\subsubsection{Influencia positiva}\label{influencia-positiva}

Para dos agentes i y j con opiniones x1, x2 respectivamente, su opinión
se modifica después de una interacción mediante la siguiente fórmula

\begin{verbatim}
let x1-new (x1 + learning-rate * (x2 - x1))
    let x2-new (x2 + learning-rate * (x1 - x2))
    set opinion x1-new
      ask other-patch [ set opinion x2-new] 
\end{verbatim}

donde el parámetro \emph{learning-rate} indica que tan rápido cambia la
opinión de un agente después de una interacción.

\subsubsection{Confianza acotada}\label{confianza-acotada}

Para dos agentes i y j con opiniones x1, x2 respectivamente, su opinión
se modifica después de una interacción mediante la siguiente fórmula

\begin{verbatim}
if (abs (x1 - x2) < confidence-threshold) [ 
      let x1-new (x1 + learning-rate * (x2 - x1))
      let x2-new (x2 + learning-rate * (x1 - x2))
      set opinion x1-new
      ask other-patch [ set opinion x2-new]
\end{verbatim}

El nuevo parámetro a considerar es \emph{confidence-threshold},
indicando la tolerancia que tiene cada agente para opiniones
disimilares. El parámetro \emph{learning-rate} actúa de la misma forma
que en el modelo de influencia positiva. De esta manera, un agente
ajusta su opinión solamente si interactua con otro agente con una
opinión lo suficientemente cercana a la que ya posee. En caso contrario,
la opinión de ambos agentes en la interacción se mantiene igual.

\subsubsection{Influencia negativa}\label{influencia-negativa}

Para dos agentes i y j con opiniones x1, x2 respectivamente, su opinión
se modifica después de una interacción mediante la siguiente fórmula

\begin{verbatim}
if (abs (x1 - x2) < confidence-threshold) [ 
      let x1-new (x1 + learning-rate * (x2 - x1))
      let x2-new (x2 + learning-rate * (x1 - x2))
      set opinion x1-new
      ask other-patch [ set opinion x2-new]
      ]
      if (abs (x1 - x2) > confidence-threshold)[ 
        ifelse (x1 > x2)
        [
          let x1-new (x1 + learning-rate * (x1 - x2) * (1 - x1) * .5)
          let x2-new (x2 + learning-rate * (x2 - x1) * (1 + x2) * .5)
          set opinion x1-new
          ask other-patch [ set opinion x2-new]
        ]
        [
          let x1-new (x1 + learning-rate * (x1 - x2) * (1 + x1) * .5)
          let x2-new (x2 + learning-rate * (x2 - x1) * (1 - x2) * .5)
          set opinion x1-new
          ask other-patch [ set opinion x2-new]
        ]
      ]
\end{verbatim}

En este caso la interacción se ve dominada por el parámetro
\emph{confidence-threshold}. Si la diferencia de las opiniones es menor
al valor de \emph{confidence-threshold}, entonces ambos ajustan su
opinión para ser más cercanas de acuerdo al valor del
\emph{learning-rate}. Sin embargo, si la diferencia entre opiniones es
mayor a el valor de \emph{confidence-threshold}, entonces ambos agentes
alejan sus opiniones de acuerdo al valor del parámetro
\emph{learning-rate}.

\subsection{Conversión de opinión a preferencia de
voto}\label{conversiuxf3n-de-opiniuxf3n-a-preferencia-de-voto}

Para convertir la opinión a preferencia de voto, se toma una opinión
mayor a cero como la preferencia por el candidato A, y una opinión
negativa como preferencia por el candidato B. De esta forma, para
actualizar las preferencias globales de los agentes presentes en el
modelo se tiene la siguiente fórmula.

\begin{verbatim}
  set pref-A (count patches with [opinion > 0] )
  set pref-B (count patches with [opinion < 0] )
\end{verbatim}

El total de agentes con opinión mayor a cero se cuenta como el total de
la población con la intención de votar por la opción A, mientra que el
total de agentes con opinión menor a cero se cuenta como el número de
agentes con la intención de votar por la opción B.

\end{document}
