 \documentclass[11pt, letterpaper]{article}

\usepackage[utf8]{inputenc}
\usepackage[T1]{fontenc}
%\usepackage{lmodern}
\usepackage{graphicx}
%\usepackage{wrapfig}
%\usepackage{rotating}
%\usepackage{subfig}
\usepackage{amsmath}
%\usepackage{textcomp}
\usepackage{amssymb}
\usepackage{hyperref}
%\usepackage{longtable}
%\usepackage{minted}
\usepackage{makecell}
\usepackage{lipsum}
\usepackage[spanish]{babel}
\usepackage[round]{natbib}

\title{\textsc{Bitácora experimental} \\
Experimentos con modelo básico de dinámica de opiniones
}  

\author{Ulises Jiménez Guerrero\\
	Alumno de la Maestría en Inteligencia Artificial \\ \\ \textbf{IIIA}
	Instituto de Investigaciones en Inteligencia Artificial \\
	Universidad Veracruzana \\ \emph{Campus Sur, Calle Paseo Lote II,
		Sección 2a, No 112} \\ \emph{Nuevo Xalapa, Xalapa, Ver., México 91097}
	\\ \\ zS24019399@estudiantes.uv.mx}

\date{\today}

\begin{document}
	\maketitle
	
\section{Introducción}
En este documento se describen los experimentos realizados sobre el modelo básico de dinámica de opiniones, juntos a los resultados obtenidos. El objetivo es lograr reproducir el cambio de opinión durante la elección presidencial en México llevada a cabo el 2 de junio de 2024. Para ello, se obtuvieron los resultados en la \href{https://prep2024.ine.mx/publicacion/nacional/presidencia/nacional/candidatura}{página web del PREP 2024}.
	
	
\section{Primer experimento}


\section{Anexo. Protocolo ODD}\label{anexo}

\end{document}